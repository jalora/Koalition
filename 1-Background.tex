\section{Background} \label{sec:background}

\subsection{Introduction}
Loyalty programs (LPs) have been implemented across the globe to develop brand loyalty by rewarding returning customers. However, current LPs fail to take full advantage of the network effects that can be gained by leveraging blockchain technology. These advantages include increasing customer loyalty, eliminating third party fees through disintermediation, providing data analytics through customer redemption patterns, and leveraging global integration to develop cooperative alliances that span industries and create a more functional LP for both consumers and merchants. 

Koalition is the next generation LP, creating a robust rewards network that integrates existing LPs and adds new opportunities for businesses that previously lacked the resources to deploy their own LP. This network of merchants provides more enticing incentives to customers by allowing them to spend their points when, where and how they want to.  Koalition enables customers to accumulate enough points to redeem for meaningful rewards without the hassle of managing separate accounts for dozens of LPs. For businesses, Koalition makes each rewards point transaction more secure and transparent, while removing trusted third party (TTP) services.

The next sections will first focus on the limitations and problems associated with current rewards programs. Then the utility of blockchain technology will be touched upon before describing how it can be used to disrupt the current loyalty industry. Furthermore, the unique aspects of Koalition?s decentralized attributes, token stability and added benefits will be described in further detail along with a roadmap for future development. 

\subsection{Types of Loyalty Programs}

\subsubsection{Stand-Alone Loyalty Programs}

Customer loyalty is an asset that can be leveraged by both small businesses and big brands. Customers loyal to a particular brand are more likely to spread positive word of mouth, less likely to be swayed by competitors, and are less price sensitive. To better compete in a competitive industry, companies in industries ranging from airlines to supermarkets to local coffee shops create incentives using LPs. These programs offer consumers rewards points proportional to their purchases and allow consumers to accumulate points to redeem different types of rewards. Many companies use LPs to increase customer retention as the cost of acquiring new customers is much greater \cite{VR16}. However, with thousands of programs to choose from, consumers oftentimes are circumscribed by making the most economic travel decisions. According to the $2015$ Colloquy Loyalty Census, the average household belongs to nearly 29 different loyalty programs \cite{CQ15}. This perpetuates frustration because customers either pay high costs for similar services to remain loyal to one LP or they have points scattered among a myriad of LPs that they are unable to redeem for anything worthwhile. The end result is account inactivity and low redemption rates. In fact, the 2016 Bond Loyalty Report which queried 12,000 Americans and 7,000 Canadians about their 280 loyalty programs across various industries, found that only 50\% of them were active members of their respective programs. Of those 50\%, a fifth of them had never redeemed their points. Unredeemed rewards points create accounting liabilities for merchants in the form of unrealized revenues that cannot be absolved until they are redeemed. Furthermore, customer retention rates diminish for LP members who do not redeem their points. These customers are 2.7 times more likely to join a different program altogether \cite{Bond16}. 

Many customers have familiarized themselves with LPs through the travel industry, however, travel companies continue to lose market share to Online Travel Agencies (OTAs). OTAs like Priceline and Expedia act as a one-stop shop for consumers by consolidating the experience of individually booking a flight, hotel, and rental car into one platform. OTAs have their own LPs which give customers more choices (so customers can select the cheapest flight with minimal opportunity costs) and make it increasingly difficult for traditional stand-alone LPs to compete. Airlines, hotels, and rental car companies share up to 20\% of their revenues with OTAs and OTAs are projected to own 41\% of travel industry bookings by 2020 \cite{OTA17}.  Despite this fact, merchants still use OTAs because exiting this channel would result in a significant loss of marketing opportunities.  It is clear that merchants need an alternative loyalty program that provides enough customer flexibility to compete with OTAs, or continue giving up revenue.

\subsubsection{Coalition Loyalty Programs}
While the number of stand-alone LPs have grown, the challenges of these programs to consumers and businesses alike have caused many businesses to search for different LP schemes. Businesses have begun to participate in coalition LPs in an effort to provide customers rewards that can be earned and redeemed faster with greater flexibility across businesses. Most coalition LPs are formed through the agglomeration of existing stand-alone LPs and consist of either pairwise partnerships such as Amex-Uber or centralized partnerships like Starwood Hotels and Resorts (U.S.) and SkyTeam (international alliance). These coalitions help split liability among the participating merchants and partially solve some of the customer frustrations while still attempting to increase customer retention and engagement rates. Furthermore, coalition LPs promote the acquisition of new customers more easily than stand-alone LPs due to inter-company cooperation which favors cross-selling. It costs less to join a coalition LP than to establish a stand-alone program, and empirical evidence shows that businesses in coalition LPs achieve higher marketing performance \cite{VR16}. 

However, coalitions are often times limited in scope. Coalitions typically form among large entities in similar industries (i.e. the many large hotel, airline, and rental car brands within the travel industry). These coalitions alienate small to medium cap businesses providing complimentary services, and hence fail to leverage the additional network benefits provided by adding these businesses. Furthermore, complex exchange rates and liability transfers between various LPs in a coalition require negotiations that consume unnecessary time and resources. While coalition LPs mark a step forward to fixing the problems that plague stand-alone LPs, these programs have a long way to go to improve customer loyalty and enhance the customer experience.

\subsubsection{Centralized Databases}
Databases are the primary means by which any digital data and applications are stored. At a high level, these databases consist of a hard drive or series of hard drives whose low level operations (i.e. where data is stored) is managed by the operating system (Windows, Mac, Linux, etc). A SQL (Structured Query Language) server runs on top of the operating system and is responsible for how the data is accessed, organized, and used. These digital databases are typically centralized and managed by either the business or a TTP. Centralized databases are easy to maintain and simple to implement, but data stored in these databases can be easily manipulated (mutable) by both external and internal actors due to its centralized nature (i.e. there is a single point of failure). Centralized databases also suffer from increased complexity and maintenance cost as the number of users increase. As users increase, the amount of data stored in the database increases, driving a need for more costly digital space, operational resources, and more IT personnel to manage the database. These problems are significantly magnified for applications that require the interchange and communication of data across different databases. These applications include the formation of consortiums or coalitions in which businesses pool their resources together to achieve a common goal. Generally, centralization introduces a significant amount of friction in these applications because it inhibits modularity (must give control to a central entity, making it difficult to change individual structure), interoperability (different technology, different data structures, etc) and inclusion (difficult, incurs a cost, or slow to add new businesses).
